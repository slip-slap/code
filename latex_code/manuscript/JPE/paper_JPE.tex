\documentclass[USenglish]{article}
\makeatletter\if@twocolumn\PassOptionsToPackage{switch}{lineno}\else\fi\makeatother

\usepackage[utf8]{inputenc}
\usepackage[T1]{fontenc}
\usepackage[small]{dgruyter} 
\let\runningauthors\runningauthor
\emergencystretch 2.5em
\makeatletter
\long\def\abstract#1{\gdef\@abstract{#1}}
\makeatother

\usepackage{amsfonts,amssymb,amsbsy,latexsym,amsmath,tabulary,graphicx,times,xcolor}


%%%%%%%%%%%%%%%%%%%%%%%%%%%%%%%%%%%%%%%%%%%%%%%%%%%%%%%%%%%%%%%%%%%%%%%%%%
% Following additional macros are required to function some 
% functions which are not available in the class used.
%%%%%%%%%%%%%%%%%%%%%%%%%%%%%%%%%%%%%%%%%%%%%%%%%%%%%%%%%%%%%%%%%%%%%%%%%%
\usepackage{url,multirow,morefloats,floatflt,cancel,tfrupee}
\usepackage{tikz,makecell,multirow,tablefootnote,threeparttable}
\usepackage{adjustbox, capt-of}
\usetikzlibrary{shapes.geometric, matrix,arrows,positioning,calc,intersections}
\makeatletter


\AtBeginDocument{\@ifpackageloaded{textcomp}{}{\usepackage{textcomp}}}
\makeatother
\usepackage{colortbl}
\usepackage{xcolor}
\usepackage{pifont}
\usepackage[nointegrals]{wasysym}
\urlstyle{rm}
\makeatletter

%%%For Table column width calculation.
\def\mcWidth#1{\csname TY@F#1\endcsname+\tabcolsep}

%%Hacking center and right align for table
\def\cAlignHack{\rightskip\@flushglue\leftskip\@flushglue\parindent\z@\parfillskip\z@skip}
\def\rAlignHack{\rightskip\z@skip\leftskip\@flushglue \parindent\z@\parfillskip\z@skip}

%Etal definition in references
\@ifundefined{etal}{\def\etal{\textit{et~al}}}{}


%\if@twocolumn\usepackage{dblfloatfix}\fi
\usepackage{ifxetex}
\ifxetex\else\if@twocolumn\@ifpackageloaded{stfloats}{}{\usepackage{dblfloatfix}}\fi\fi

\AtBeginDocument{
\expandafter\ifx\csname eqalign\endcsname\relax
\def\eqalign#1{\null\vcenter{\def\\{\cr}\openup\jot\m@th
  \ialign{\strut$\displaystyle{##}$\hfil&$\displaystyle{{}##}$\hfil
      \crcr#1\crcr}}\,}
\fi
}

%For fixing hardfail when unicode letters appear inside table with endfloat
\AtBeginDocument{%
  \@ifpackageloaded{endfloat}%
   {\renewcommand\efloat@iwrite[1]{\immediate\expandafter\protected@write\csname efloat@post#1\endcsname{}}}{\newif\ifefloat@tables}%
}%

\def\BreakURLText#1{\@tfor\brk@tempa:=#1\do{\brk@tempa\hskip0pt}}
\let\lt=<
\let\gt=>
\def\processVert{\ifmmode|\else\textbar\fi}
\let\processvert\processVert

\@ifundefined{subparagraph}{
\def\subparagraph{\@startsection{paragraph}{5}{2\parindent}{0ex plus 0.1ex minus 0.1ex}%
{0ex}{\normalfont\small\itshape}}%
}{}

% These are now gobbled, so won't appear in the PDF.
\newcommand\role[1]{\unskip}
\newcommand\aucollab[1]{\unskip}
  
\@ifundefined{tsGraphicsScaleX}{\gdef\tsGraphicsScaleX{1}}{}
\@ifundefined{tsGraphicsScaleY}{\gdef\tsGraphicsScaleY{.9}}{}
% To automatically resize figures to fit inside the text area
\def\checkGraphicsWidth{\ifdim\Gin@nat@width>\linewidth
	\tsGraphicsScaleX\linewidth\else\Gin@nat@width\fi}

\def\checkGraphicsHeight{\ifdim\Gin@nat@height>.9\textheight
	\tsGraphicsScaleY\textheight\else\Gin@nat@height\fi}

\def\fixFloatSize#1{}%\@ifundefined{processdelayedfloats}{\setbox0=\hbox{\includegraphics{#1}}\ifnum\wd0<\columnwidth\relax\renewenvironment{figure*}{\begin{figure}}{\end{figure}}\fi}{}}
\let\ts@includegraphics\includegraphics

\def\inlinegraphic[#1]#2{{\edef\@tempa{#1}\edef\baseline@shift{\ifx\@tempa\@empty0\else#1\fi}\edef\tempZ{\the\numexpr(\numexpr(\baseline@shift*\f@size/100))}\protect\raisebox{\tempZ pt}{\ts@includegraphics{#2}}}}

%\renewcommand{\includegraphics}[1]{\ts@includegraphics[width=\checkGraphicsWidth]{#1}}
\AtBeginDocument{\def\includegraphics{\@ifnextchar[{\ts@includegraphics}{\ts@includegraphics[width=\checkGraphicsWidth,height=\checkGraphicsHeight,keepaspectratio]}}}

\DeclareMathAlphabet{\mathpzc}{OT1}{pzc}{m}{it}

\def\URL#1#2{\@ifundefined{href}{#2}{\href{#1}{#2}}}

%%For url break
\def\UrlOrds{\do\*\do\-\do\~\do\'\do\"\do\-}%
\g@addto@macro{\UrlBreaks}{\UrlOrds}



\edef\fntEncoding{\f@encoding}
\def\EUoneEnc{EU1}
\makeatother
\def\floatpagefraction{0.8} 
\def\dblfloatpagefraction{0.8}
\def\style#1#2{#2}
\def\xxxguillemotleft{\fontencoding{T1}\selectfont\guillemotleft}
\def\xxxguillemotright{\fontencoding{T1}\selectfont\guillemotright}

\newif\ifmultipleabstract\multipleabstractfalse%
\newenvironment{typesetAbstractGroup}{}{}%

%%%%%%%%%%%%%%%%%%%%%%%%%%%%%%%%%%%%%%%%%%%%%%%%%%%%%%%%%%%%%%%%%%%%%%%%%%

\usepackage[numbers,sort&compress]{natbib}

\usepackage{float}

\begin{document}

\def\authorCount{2}
\def\affCount{1}

\def\journalTitle{Journal of Polymer Engineering}

\title{A technique for constrained optimization of symmetric laminate using a new variant of genetic algorithm}
\author[]{Huiyao Zhang}
\author*[]{Atsushi Yokoyama}
\affil[1]{Department of Fiber Science and Engineering\unskip, Kyoto Institute of Technology,\unskip,
Kyoto\unskip, 606-8585\unskip, Kyoto\unskip, Japan; Email: mailto:yokoyama@kit.ac.jp}


\abstract{
 The main challenge presented by the laminate composite design is the laminate layup, involving a
 set of fiber orientation, composite material system, and stacking sequence. In nature, it is a
 combinatorial optimization problem that can be solved by the genetic algorithm(GA). In this present
 study, a new variant of GA is introduced for the optimal design by modifying the selection
 strategy. To improve this new GA's performance, a particular group is maintained in the population
 during the optimization process. To check the feasibility of a laminate subject to in-plane
 loading, the effect of fiber orientation angles and materials component on the first ply failure is
 studied. A comparative study of basic GA and an improved GA in laminate composite designing for a
 targeted safety factor is also studied. An optimal composite material and laminate layup is
 well-developed for a targeted strength ratio, which compromises weight and cost through an improved
 genetic algorithm. Numerical results are obtained and presented for different loading cases.
}\def\keywordstitle{Keywords}\runningauthors{Zhang and Yokoyama}
\aop 
\maketitle 
    
\section{Introduction}
Composite materials offer improved strength, stiffness, fatigue, corrosion resistance, etc. over
conventional materials, widely used as materials ranging from automotive to the shipbuilding
industry, electronic packaging to golf clubs, and medical equipment homebuilding. However, the high
cost of fabrication of composites is a critical drawback for its application. For example, the
graphite/epoxy composite part may cost as much as $650 to $900 per kilogram. In contrast, the price
of glass/epoxy is about 2.5 times less. Manufacturing techniques such as sheet molding compound and
structural reinforcement injection molding are taken to lower manufacturing automobile parts' cost.
An alternative approach is using hybrid composite material.

  The mechanical performance of a laminate composite is affected by a wide range of factors,
thickness, material ,and orientation of each lamina. Because of manufacturing limitation, all these
variables are usually limited to a small set of discrete values. For example, ply thickness is fixed
and ply orientation angles are limited to a set of angles such as 0,45,90 degrees in practice. So
the search process for the optimal design is a discrete optimization problem which can be solved by
GA. To tailor a laminate composite, GA has been successfully applied to solve laminate design
problem\cite{riche1993optimization,nagendra1996improved,sadagopan1998application,todoroki1998stacking,liu2000permutation,sivakumar1998optimum,walker2003technique,lin2004stacking,kang2005minimum,murugan2007target,akbulut2008optimum}.
GA simulates the process of natural evolutionary includes selection, crossover, and mutation
according to Darwin's principal of '' survival of the fittest''. The known advantage of GA as the
following:(i): GAs are not easily trapped in local optimum, and be able to obtain the global
optimum. (ii): GA doesn't need gradient information and can be applied to discrete optimization
problem. (iii): GA not only be able to find the optimal value in the domain, but also can maintain a
set of optimal solutions. On the other hand, GA also has some disadvantages, for example, the GA
needs to evaluate the target functions a lot of times to achieve the optimization, the cost of the
search process is high. The GA consists of some basic parts, the coding of the design variable,
selection strategy, crossover operator, mutation operator, and how to deal with constraints. For the
variable design part, there are two methods to deal with the representation of design variables,
binary string and real value representation\cite{riche1993optimization,todoroki1998stacking}.
Michalewicz\cite{zbigniew1996genetic} claimed the performance of floating-point representation was
better than binary representation in numerical optimization problem. Selection strategy plays a
critical role in GA which decides the convergence speed and the diversity of the population. To
improve search ability and reduce search cost, various selection methods have been invented, and it
can be divided into four classes which are proportionate reproduction, ranking, tournament, and
genitor(or '' steady state'') selection. In the optimization of laminate composite design, roulette
wheel\cite{riche1993optimization,seresta2007optimal}, where the possibility of an individual to be
chosen for the next generation is proportional to its fitness.
Soremekun\cite{soremekun2001composite} showed that generalized elitist strategy outperformed a
single individual elitism in some special cases.

  Data structure, repair strategies and penalty functions\cite{le1995improved} are most common used
approaches to resolve constrained problems in the optimization of composite structure. Symmetric
laminates are widely used in practical scenario, data structure can be used to fulfil the symmetry
constraint which consists of coding a half of the laminate and considering the rest with the
opposite orientation. Todoroki\cite{todoroki1998stacking} introduced a repair strategy which can scan the chromosome and
repair the gene on the chromosome if it doesn't satisfy the contiguity constraint. The comparsion of
repair strategies in a permutation GA with the same orientation was presented by Liu\cite{liu2000permutation}, and it
showed the Baldwinian repair strategy can significantly reduce the cost of constrained optimization.
Haftka\cite{riche1993optimization} used GA to solve laminate stacking sequence problem using a penalty function subject to
buckling and strength constraints.

  In typical engineering applications, composite materials are under very complicate loading
conditions, not only inplane loading but also out-of-plane loading. Most of the studies on the
optimization of laminate composite material was to mimimize the
thickness\cite{abu1998optimum,walker2003technique},
weight\cite{fang1993design,deka2005multiobjective,park2008improved}, cost and
weight\cite{deka2005multiobjective,omkar2008artificial}, or maximize the static strength of
composite laminates for a targeted
thickness\cite{walker2003technique,lin2004stacking,kim2007development}. In the present study,
laminate cost and weight are minimized by modifying the objective function.

  In order to check the feasibility of a laminate composite by imposing a strength constraint, failure
analysis of a laminate is taken by applying suitable failure criteria. The failure criteria of
laminated composites can be classified in three classes: non-interactive theories(e.g., Maximum
strain), interactive theories(e.g., Tsai-wu), and partially interactive theories(e.g., Puck failure
criterion). The previous researchers adopted the first-ply-failure approach using the Tsai-wu
failure
theory\cite{massard1984computer,reddy1987first,fang1993design,soeiro1994multilevel,pelletier2006multi,jadhav2007parametric,omkar2008artificial,choudhury2019failure},
Tsai-Hill\cite{martin1987optimum,soares1995discrete}, the maximum stress\cite{watkins1987multicriteria}, or the maximum strain\cite{watkins1987multicriteria}
static failure criteria. Akbulut\cite{akbulut2008optimum} used GA to minimize the thickness of composite laminates with
Tsai-Hill and maximum stress failure criteria, and the advantage of this method is to avoid spurious
optima. Naik\cite{naik2008design}
 minimized the weight of laminated composites under restrictions with a
failure-mechanism based criterion based on Maximum Strain and Tsai-wu. In the present study, Tsai-wu
static failure criteria is used to investigate  the feasibility of a laminate composite.
    
\section{Stress and Strain in a Laminate}
A laminate structure consists of multiple lamina bonded together through their thickness.
Considering a laminate composite plate which is symmetric to its middle plane subject to in-plane
loading of extension, shear, bending and torsion, the classical lamination theory(CLT) is taken to
calculate the stress and strain in the local and global axes of each ply, as shown in 
Fig.\ref{fig:lamina}.


\begin{figure}
	\centering
	\includegraphics[width=\linewidth]{A_laminate_design_images/lamina_local_global_axes.png}
	\caption{Lamina}
  	\label{fig:lamina}
\end{figure}



\subsection{Stress and Strain in a Lamina}
For a single lamina, the stress strain relation in the local axis 1-2:
\begin{equation}
    \begin{bmatrix}
        \sigma _1\\
        \sigma _2\\
        \tau_{12}
    \end{bmatrix}
    =
    \begin{bmatrix}
        Q_{11} & Q_{12} & 0\\
        Q_{12} & Q_{22} & 0\\
        0      &  0     & Q_{66}
    \end{bmatrix}
    \begin{bmatrix}
        \varepsilon_1\\
        \varepsilon_2\\\gamma_{12}
    \end{bmatrix}
\end{equation}
Where $Q_{ij} $are the stiffnesses of the lamina that are related

to engineering elastic constants given by
\begin{equation}
    \begin{split}
    &Q_{11}=\frac{E_1}{1-v_{12}v_{21}}\\
    &Q_{22}=\frac{E_2}{1-v_{12}v_{21}}\\
    &Q_{66}=G_{12}\\
    &Q_{12}=\frac{v_{21}E_2}{1-v_{12}v_{21}}\\
    \end{split}
\end{equation}

Where $E_1, E_2, v_{12}, G_{12} $  are four independent engineering elastic constants, they are defined as, $E_1 $  is  longitudinal Young's modulus, $E_2 $  is transverse Young's modulus, $v_{12} $ is  major Poisson's ratio $G_{12} $ is in-plane shear modulus.

Stress strain relation in global axis x-y:
\begin{equation}\left[\begin{array}{l}\sigma _{x} \\ \sigma _{y} \\ \tau_{xy}\end{array}\right]=\left[\begin{array}{lll}\bar{Q}_{11} & \bar{Q}_{12} & \bar{Q}_{16}\\ \bar{Q}_{12} & \bar{Q}_{22} & \bar{Q}_{26} \\ \bar{Q}_{16} & \bar{Q}_{26} &\bar{Q}_{66}\end{array}\right]\left[\begin{array}{l}\varepsilon_{x} \\ \varepsilon_{y}\\ \gamma_{x y}\end{array}\right]
\end{equation}
where
\begin{equation}\begin{array}{l}\bar{Q}_{11}=Q_{11} c^{4}+Q_{22} s^{4}+2\left(Q_{12}+2 Q_{66}\right) s^{2} c^{2}\\ \bar{Q}_{12}=\left(Q_{11}+Q_{22}-4 Q_{66}\right) s^{2} c^{2}+Q_{12}\left(c^{4}+s^{2}\right)\\ \bar{Q}_{22}=Q_{11} s^{4}+Q_{22} c^{4}+2\left(Q_{12}+2 Q_{66}\right) s^{2} c^{2} \\\bar{Q}_{16}=\left(Q_{11}-Q_{12}-2 Q_{66}\right) c^{3} s-\left(Q_{22}-Q_{12}-2Q_{66}\right) s^{3} c \\ \bar{Q}_{26}=\left(Q_{11}-Q_{12}-2 Q_{66}\right) c s^{3}-\left(Q_{22}-Q_{12}-2 Q_{66}\right)c^{3} s \\ \bar{Q}_{66}=\left(Q_{11}+Q_{22}-2 Q_{12}-2 Q_{66}\right) s^{2}c^{2}+Q_{66}\left(s^{4}+c^{4}\right)\\\end{array}
\end{equation}

The c and s denotes $cos\theta $ , and $sin\theta $ .

The local and global stresses in an angle lamina are related

to each other through the angle of lamina $\theta $
\begin{equation}\left[\begin{array}{l}\sigma _{1} \\ \sigma _{2} \\ \tau_{12}\end{array}\right]=[T]\left[\begin{array}{l}\sigma _{x} \\ \sigma _{y} \\\tau_{xy}\end{array}\right]
\end{equation}

where
\begin{equation}[T]=\left[\begin{array}{ccc}c^{2} & s^{2} & 2 s c \\ s^{2} & c^{2} & -2 s c \\ -s c & s c &c^{2}-s^{2}\end{array}\right] 
\end{equation}





\subsection{Stress and Strain in a Laminate}

\begin{equation} \label{eq:force_and_moments}
	\begin{array}{l}
	\begin{bmatrix}
		N_x \\
		N_y \\
		N_{xy}
	\end{bmatrix}
	=
	\begin{bmatrix}
		A_{11} & A_{12} & A_{16} \\
		A_{12} & A_{22} & A_{26} \\
		A_{16} & A_{26} & A_{66} 
	\end{bmatrix}
    \begin{bmatrix}
		\varepsilon_x^0 \\
        \varepsilon_y^0 \\
		\gamma_{xy}^0
    \end{bmatrix}  
	+              
	\begin{bmatrix}
		B_{11} & B_{12} & B_{16} \\
		B_{11} & B_{12} & B_{16} \\
		B_{16} & B_{26} & B_{66} 
	\end{bmatrix}
	\begin{bmatrix}
		k_x \\
		k_y \\
		k_{xy} 
	\end{bmatrix}  \\
	\\

	\begin{bmatrix}
		M_x \\
		M_y \\
		M_{xy}
	\end{bmatrix}
	=
	\begin{bmatrix}
		B_{11} & B_{12} & B_{16} \\
		B_{12} & B_{22} & B_{26} \\
		B_{16} & B_{26} & B_{66} 
	\end{bmatrix}
    \begin{bmatrix}
		\varepsilon_x^0 \\
        \varepsilon_y^0 \\
		\gamma_{xy}^0
    \end{bmatrix} 
	+ 
	\begin{bmatrix}
		D_{11} & D_{12} & D_{16} \\
		D_{11} & D_{12} & D_{16} \\
		D_{16} & D_{26} & D_{66} 
	\end{bmatrix}
	\begin{bmatrix}
		k_x \\
		k_y \\
		k_{xy} 
	\end{bmatrix}
	\end{array}
\end{equation}

$N_x,N_y $  - normal force per unit length

$N_{xy} $  - shear force per unit length

$M_x, M_y $ - bending moment per unit length

$M_{xy} $  - twisting moments per unit length

$\varepsilon^{0}, k $- mid plane strains, and curvature of laminate in x-y coordinate

The mid plane strain and curvature is given by
where 
\begin{equation}
    \begin{split}
    &A_{ij}=\sum_{k=1}^{n}(\overline{Q_{ij}})_k(h_k-h_{k-1})  i=1,2,6, j=1,2,6\\
    &B_{ij}=\frac{1}{2}\sum_{k=1}^{n}(\overline{Q_{ij}})_k(h_k-h_{k-1})  i=1,2,6, j=1,2,6\\
    &D_{ij}=\frac{1}{3}\sum_{k=1}^{n}(\overline{Q_{ij}})_k(h_k-h_{k-1}) i=1,2,6, j=1,2,6\\
    \end{split}
\end{equation}

The [A], [B], and [D] matrices are called the extensional, coupling, and bending stiffness matrices.

    
\section{Failure Theory}
Many different theories about the failure of an angle lamina have been developed for a
unidirectional lamina, such as maximum stress failure theory, maximum strain failure theory,
Tsai-Hill failure theory, and Tsai-Wu failure theory. The failure theories of a lamina are based on
the stresses in local axes in the material. There are four normal strength parameters and one shear
stress for a unidirectional lamina. The five strength parameters are  

$(\sigma _1^{T})_{ult}= $Ultimate longitudinal tensile strength,

$(\sigma _1^{C})_{ult}= $ Ultimate longitudinal compressive strength,

$(\sigma _2^{T})_{ult}= $ Ultimate transverse tensile strength,

$(\sigma _2^{C})_{ult}= $Ultimate transverse compressive strength, and

$(\tau_{12})_{ult}= $ Ultimate in-plane shear strength

  In the present study, Tsai-wu failure theory is taken to decide whether a lamina is failed or not,
the reason is this theory is more general than Tsai-Hill failure theory which considers two
different situation, compression and tensile strength of a lamina. A lamina is considered to be
failed if \begin{equation} \label{eq:tsai_wu}
\begin{split}
	H_1 \sigma_1  + H_2 \sigma_2 + H_6 \tau_{12} + H_{11}\sigma_1^2 + H_{22} \sigma_2^2 
                  + H_{66}  \tau_{12}^2 + 2H_{12}\sigma_1\sigma_2 < 1
\end{split}
\end{equation}

is violated. where

\begin{equation} \label{eq:sr}S R=\frac{\text {Maximum Load Which Can Be Applied}}{\text {Load Applied}}
\end{equation}


\subsection{Failure Theories of a Laminate}
A laminate will fail under increasing mechanical, however, the procedure of laminate failure may not
be catastrophic. In some cases, some layer fail first and the rest is able to continue to take more
loads until all the plies fail. A ply is fully discount when a ply fails, then the ply is replaced
by near zero stiffness and strength. The procedure for finding the first ply failure in the present
study follows the fully discounted method4:

\begin{enumerate}
	\item Compute the reduced stiffness matrix [Q] referred to local axis for each ply using its four engineering elastic constants $E_1 $, $E_2 $, $E_{12} $, and $G_{12} $.

	\item Calculate the transformed reduced stiffness $[\bar{Q}] $ referred to global coordinate system (x, y) using reduced stiffness matrix [Q] obtained in step 1 and ply angle for each layer.

	\item  Given the thickness and the location of each layer, and find out the three laminate stiffness matrices [A], [B], and [D].

	\item  Apply forces and moments, $[N]_{xy}, [M]_{xy} $ solve the
		Equation\ref{eq:force_and_moments} , calculate the middle plane strain $[\sigma ^{0}]_{xy} $ and curvature $[k]_{xy} $.

	\item Find out the local strain and stress of each layer under the applied load.

	\item  Use the ply-by-ply stresses and strains in Tsai-wu failure theory to find out the strength ratio.
\end{enumerate}



    
\section{Optimum Design of Laminate Composite}


\subsection{Genetic Algorithm}
The GA starts off with a bunch of individuals with limited chromosome length, in which maybe none of
these individuals fulfill the safety factor constraint. The GA is supposed to derive appropriate
offspring based on initial population as the GA goes on. The classic way to handle constrained
search of GA are either introducing repair strategies or using a penalty function, here a new
approach is come up with to deal with constrained GA search problem by modifying the selection
strategy.

  Because of the existence of constraint, it means that the population not only can be sorted by
fitness(which is obtained by objective function), but also can sorted by the constraint value
obtained by the constraint function(assuming constraint function exists), so the parents of next
generation can be chosen by the following two approaches. First, sort out the population by the
absolute difference between the individual's constraint value and the threshold of constraint in an
ascending order, and individual with smaller difference is more likely to be chosen. Individuals
obtained by this method are called potential individuals. Second, sort out the population by fitness
from low to high after remove the improper individuals, and an individual is proper which means it
fulfills the constraint, and individuals obtained by this way are called proper individuals. So the
final parents consists of two parts, potential individuals and proper individuals, and the number of
potential individuals and proper individual are called, respectively, potential number and proper
number. For example, assuming the parent population is 20, 60 percent of them is potential
individuals, and the rest is proper individuals. So the potential number is 12, and the proper
number is 8.

  At the beginning of the GA, no individual in the population is appropriate, which means the number
of proper individuals nearly zero. So the GA can be divided into two stages according to whether
proper individual are generated during the search process. During the initial stages, the number of
potential individuals gradually decreases from maximum(which is parent population) to the potential
number, while the number of proper individuals increases from zero to the proper number as the GA
goes on. After the initial stage, both of the number of two groups, respectively, converge to
potential number, and proper number. In order to differentiate the current selection methods from
the following, the current GA is called basic GA. In the following experiment, 50 percent of the
parent are potential individuals, and 50 percent of the parent are proper individuals.

  The problem with this basic GA is premature and weak local search ability, basic GA are more likely
to get stuck in local optimum. Therefore, to prevent the GA from early convergence and improve the
local search performance, a new selection method is proposed, which is ignoring whether the
individual satisfy the constraint or not, and ranking individuals by their fitness. Individuals
selected by this method are called active individuals, because they are supposed to be always in the
population. GA with this active individuals are called improved GA. In the improved GA, parents
consists of three parts: active individuals, potential individuals, and proper individuals. In the
following experiment, 20 percent of the parent population are active individuals, 30 percent of the
parent are potential individuals, and the rest is proper individuals.

  In the present study, the relevant parameters of GA are as shown in Table \ref{tab:ga}. The design
variables are the materials, number of layers, and ply orientation restricted to a discrete set of
angles ($0,\pm 45 \text{ and } 90 \text{ degrees} $). The possible materials are graphite/epoxy,
carbon/epoxy, and glass/epoxy and are represented by the codes 0, 1 and 2, respectively.


\captionof{table}{GA parameters}
\begin{adjustbox}{width={\textwidth},totalheight={\textheight},keepaspectratio}
\begin{tabular}{ccccccc}
	\toprule
	Parameter &  Seed &Population size & LRIC  & Encoding &  Crossover Strategy& Mutation strategy\\
	\midrule
	Value     & 1     &10               & [3-15]& Integer  &  One-point &Mass mutation   \\
	\bottomrule
\end{tabular}
\end{adjustbox}
\label{tab:ga}
\begin{tablenotes}\footnotesize
\item{"LRIC" denotes the length range of initial chromosome.}
\end{tablenotes}

    
  The laminate chromosome is  represented by a double-gene string which can be divided into two parts,
one part represents the angles, the other part represents the materials(as shown in Figure
\ref{GA:operator}($P_1$)) . To maintain the diversity of the population, single-point crossover is
taken during the evolution process. The break point in the string are chosen randomly, and one of
the offspring of parent 1(as shown in Figure \ref{GA:operator}($P_1$))  and parent 2(as shown in
Figure \ref{GA:operator}($P_2$)) is obtained by combining the gene segments $P1_o$ and $P2_o$,
$P1_m$ and $P2_m$, respectively. The gene code of the offspring laminate is
$[\text{+}45,\text{-}45,\text{-}45,\text{-}45,\text{-}45,\text{-}45,\text{-}45,0,1,0,1,1,0,1,0]$.

\begin{figure}
  \includegraphics[width=\linewidth]{A_laminate_design_images/ga_operator.png}
\caption{GA Operators\label{GA:operator}}
\end{figure}

  To prevent the search from getting stuck in a local optimum, mutation is used to random change the
gene in the chromosome, the offspring after mutation operator is as shown in Figure
\ref{GA:operator}

  The GA is a stochastic procedure which heavily depends on the generator of pseudo random numbers. In
the present study, the standard Wichmann-Hill generator is used in the algorithm, which combines
three pure multiplicative congruent generators of modulus 30269, 30307 and 30323.  The seed used
in this paper is 1.

\subsection{Design Problem I}

The aim is to minimize the mass of a laminate composite for a targeted strength
ratio by Tsai-wu failure theory. The design variable are the ply angles, and the
number of layers.

Find: $\{\theta_k, n\}$ $\theta_k \in \{ 0,\text{+}45,\text{-}45,90\}$ 

Minimize: weight

Subject to: strength ratio and first ply failure constraint


\subsection{Design Problem II}
The aim is to minimize the weighted cost and weight of hybrid composite
laminate under various loading cases, so the design variable not only include
the ply angles and number of layers, but also the material of each lamina. 


Find: $\{\theta_k,\text{mat}_k, n\}$ $\theta_k \in \{ 0,\text{+}45,\text{-}45,90\}$ $\text{mat}_k \in \{CA, GR, GL \}$

Minimize: 
\begin{equation}
	F=\frac{\text { Cost }}{C_{\text {min }}}+\frac{\text { Weight }}{W_{\text {min }}}
\end{equation}

Subject to: strength ratio and first ply failure constraint


Here CA, GF, and GL represent carbon/epoxy, graphite/epoxy, and glass/epoxy,
 $C_{\text{min}}$ and $W_{\text{min}}$ represent the cost and
weight corresponding to the laminates with minimum cost and minimum weight
obtained from previous problem.

\section{Numberical Results and Discussion}
A laminate composite with dimensions $1000 \times 1000 \times 0.165 mm^3$ of
each lamina is under various loading cases, and each CA, GF, and GL layer is
assumed to cost 8, 2.5 and 1 monetary units, respectively.  The other used
material properties are as shown in Table \ref{tab:mat}.  In the present
experiment,  the optimal composite system, layup, thickness, and number of
layers for a targeted strength ratio(2 in this paper) under two different
in-plane loading is investigated.


\captionof{table}{Comparsion of carbon/epoxy, graphite/epoxy, and glass/epoxy properties}
\begin{adjustbox}{width={\textwidth},totalheight={\textheight},keepaspectratio}
\begin{tabular}{cccccc}
	\toprule
	Property								   & Symbol				  & Unit  &  Carbon/Epoxy&  Graphite/Epoxy  &  Glass/Epoxy   \\
	\midrule																								  
	Longitudinal elastic modulus			   & $E_1$				  & GPa   &  116.6       &  181             &  38.6           \\
	Traverse elastic modulus				   & $E_2$				  & GPa   &  7.67        &  10.3            &  8.27           \\
	Major Poisson's ratio					   & $v_{12}$			  &       &  0.27        &  0.28            &  0.26           \\
	Shear modulus							   & $G_{12}$			  & GPa   &  4.17        &  7.17            &  4.14           \\
	Ultimate longitudinal tensile strength     & $(\sigma_1^T)_{ult}$ & MP    &  2062        &  1500            &  1062            \\
	Ultimate longitudinal compressive strength & $(\sigma_1^C)_{ult}$ & MP    &  1701        &  1500            &  610             \\
	Ultimate transverse tensile strength       & $(\sigma_2^T)_{ult}$ & MPa   &  70          &  40              &  31              \\
	Ultimate transverse compressive strength   & $(\sigma_2^C)_{ult}$ & MPa   &  240         &  246             &  118              \\
	Ultimate in-plane shear strength           & $(\tau_{12})_{ult}$  & MPa   &  105         &  68              &  72               \\
	Density                                    & $\rho$               & $g/cm^3$ &  1.605    &  1.590           &  1.903               \\
	Cost                                       &                      &       &  8           &  2.5             &  1               \\
	\bottomrule
\end{tabular}
\end{adjustbox}
\label{tab:mat}

\begin{figure}
  \includegraphics[width=\linewidth]{A_laminate_design_images/NxNy.png}
  \captionof{figure}{GA process under load  $N_x=N_y$ = 1e6 N}
  \label{fig:NxNy}
\end{figure}

\begin{table}
\caption{The optimum lay-ups for the loading $N_x=N_y=1e6$ N}
\begin{tabular}{ccccccc}
	\toprule
	 Problem  &   Algorithm      & Stacking sequence                                    & Strength ratio  & Mass  &  Cost   & Layer    \\ 
	\midrule																								  
	      I   &  GA   &  $[\text{-}45_{6}^{cr}/\text{+}45_{6}^{cr}]_s$                        & 2.026           & 1.271 &  192.0  & 24  \\
	      I   &  IGA   &  $[\text{-}45_{6}^{cr}/\text{+}45_{6}^{cr}]_s$                        & 2.026           & 1.271 &  192.0  & 24  \\
	      I   &  GA    &  $[0_6^{gr}/\text{-}45_{4}^{gr}/\text{+}45_{4}^{gr}/90_{7}^{gr}/\bar{90}^{gr}]_s$     & 2.051           & 2.256 &  107.5  & 43  \\
	      I   &  IGA    &  $[\text{+}45_{10}^{gr}/\text{-}45_{10}^{gr}/\bar{\text{-}45}^{gr}]_s$    & 2.024           & 2.151 &  102.5  & 41  \\
	      I   &  GA    &  $[\text{-}45_{35}^{gl}/\text{+}45_{36}^{gl}/\bar{\text{+}45}^{gl}]_s$    & 2.001           & 8.980 &  143.0  & 143  \\
	      I   &  IGA    &  $[\text{-}45_{35}^{gl}/\text{+}45_{36}^{gl}/\bar{\text{+}45}^{gl}]_s$    & 2.001           & 8.980 &  143.0  & 143  \\
	      II  &  GA    &
	$[\text{-}45_{12}^{gr}/\text{+}45_{5}^{cr}/\text{+}45_{7}^{gr}]_s$         & 2.141
										  & 2.523 & 175& 48  \\
	      II  &  IGA    &
	$[\text{-}45_{9}^{gr}/\text{+}45_{9}^{gr}/\text{-}45_{2}^{cr}/\text{+}45_{2}^{cr}]_s$         & 2.054
										  & 2.313 & 154& 44  \\
	\bottomrule
\end{tabular}
\label{tab:NxNy}
\end{table}
\begin{tablenotes}\footnotesize
\item{Subscript "cr" denotes a carbon/epoxy ply, "gr" denotes graphite/epoxy ply, "gl" denotes a
	glass/epoxy ply}
\end{tablenotes}

  The GA process can be divided into two phases by whether there are individuals which are appropriate
or not. During the initial phase, no individual's strength ratio is over the specified threshold, so
individuals with bigger fitness are more likely to be chosen as parents, that's why the strength
ratio curves go all the way up to the specified threshold during the first stage; After the initial
phase, the GA produces a bunch of appropriate individuals, and then the target function comes to
play, as you can see from Fig.\ref{fig:NxNy}, the fitness curves are trending to go down, but the
strength ratio curves are keep to greater the specified threshold.


  In the first experiment, the applied stress are $N_x=N_y=1e6$ N.  As shown in the
Figure\ref{fig:NxNy}, the Figure \ref{fig:NxNy}(a), (b), and (c) were the experiment results for single material,
Figure \ref{fig:NxNy}(d) is for hybrid composite material. For the single materials, both of the basic GA and improved
GA method obtained the optimal value, but the improved GA converged more slowly than the basic GA.
As it can be seen from Table \ref{tab:NxNy}, a $[\text{-}45_{6}/\text{+}45_{6}]_s$ carbon/epoxy
laminate has the least weight, denoted by $W_{min}$, and a
$[\text{-}45_{35}/\text{+}45_{73}/\text{+}45_{35}]$ graphite/epoxy laminate has the lowest cost,
denoted by $C_{min}$. The $W_{min}$ and $C_{min}$ were used to evaluate the fitness of the second
problem, which is the layup design of hybrid composite material , as shown in sub-figure d,the
improved GA obtained a more appropriate system layup, whose strength ratio is greater than the
specified safety factor, and weight and cost is less than the result obtained by basic GA method, as
shown in Table \ref{tab:NxNy}. Compared with basic GA, the improved GA method showed more powerful
global search ability in the initial phase.

\begin{figure}
  \includegraphics[width=\linewidth]{A_laminate_design_images/NxNyNz.png}
  \captionof{figure}{GA process under load  $N_x=N_y=N_z$ = 1e6 N}
  \label{fig:NxNyNz}
\end{figure}

\begin{table}
	\caption{The optimum lay-ups for the loading $N_x=N_y=N_z$ = 1e6 N} 
	\begin{tabular}{ccccccc}
	\toprule
	       Problem  &   Algorithm      & Stacking sequence                                    & Strength ratio  & Mass  &  Cost   & Layer    \\ 
	\midrule																								  
	   	  I  & GA   &  $[\text{+}45_{11}^{cr}/\text{-}45^{cr}/\bar{\text{+}45}^{cr}]_s$                            & 2.018           & 1.324 &  200.0  & 25  \\
	      I  & IGA  &  $[\text{+}45_{6}^{cr}]_s$                            & 2.041           & 0.636 &  96.0  & 12  \\
          I  & GA   &  $[0_4^{gr}/\text{+}45_{12}^{gr}/90_3^{gr}/\bar{\text{+}45}]_s$                            & 2.001           & 2.046 &  97.5  & 39  \\
		  I  & IGA  &  $[\text{+}45_{9}^{gr}]_s$                            & 2.227           & 0.945 &  45.0  & 18  \\
		  I  & GA   &  $[\text{+}45_{11}^{gl}/\bar{\text{+}45}^{gl}]_s$                            & 2.015           & 1.444 &  23.0  & 23  \\
		  I  & IGA  &   $[\text{+}45_{11}^{gl}/\bar{\text{+}45}^{gl}]_s$                           & 2.015           & 1.444 &  23.0  & 23  \\
		  II &  GA  &  $[\text{+}45^{gl}/\text{+}45_{8}^{gr}]_s$          & 2.031           & 0.965 &  42.0  & 18 \\
		  II & IGA  &  $[\text{+}45_8^{gr}/\bar{\text{+}45}^{gl}]_s$          & 2.005           & 0.902 &  41.0  & 17 \\
	\bottomrule
\end{tabular}
\label{tab:NxNyNz}
\end{table}

  In the second case, the applied stress were $N_x=N_y=N_z=1e6$ N, the experiment results were as
shown in the Figure \ref{fig:NxNyNz}. In the first experiment, as can be seen from Figure \ref{fig:NxNyNz}(a), the improved GA got a
better system layup then result obtained by basic GA; In the second experiment, as shown in the Figure\ref{fig:NxNyNz}(b), during
the initial phase, the fitness curves of basic GA and improved GA went all the way up to the
previous specified threshold, however, the improved GA converged more slowly then the basic GA which
means the search cost of improved GA is greater than basic GA. After the initial phase, the fitness
curve of basic didn't change anymore, it got trapped in local.  However, the fitness curve of
improved GA was gradually going down, at the same time,  the strength ratio curve of improved GA
were keep to be greater than the threshold. It means the improved GA was able to get out of optimum
and obtained a much better system layup.  The improved GA offered more powerful local search
ability. In the third experiment, as shown in Figure \ref{fig:NxNyNz}, both of basic GA and improved
GA obtained the same result, but the improved GA converged more slowly than the basic GA. From these
three experiments for single material, we knew a $[\text{+}45_{6}^{cr}]_s$ carbon/epoxy laminate has
the least mass, and a $[\text{+}45_{11}^{gl}/\bar{\text{+}45}^{gl}]_s$ glass laminate has the least
cost. In the last experiment, the improved GA obtained a little bit better result than the basic GA,
as it shown in Table \ref{tab:NxNyNz}, Compared with the $[\text{+}45_{12}^{cr}]$ laminate, the
weight of a $[\text{+}45_8^{gr}/\bar{\text{+}45}^{gl}]_s$ laminate increases $41.8\%$, however, the
cost decreases $56\%$.


\section{Conclusions}
In this paper, a combination of CLT and a variant of GA are employed to minimize the weight and cost
of a single-material and hybrid composite laminate, respectively, under various in-plane loading
cases.  Results are presented in two sections, stacking sequence optimization for a single material
laminate, and weighted  mass and cost optimization of a carbon/epoxy, graphite/epoxy, and
glass/epoxy hybrid laminate.  Furthermore, the performance of the basic GA is improved by changing the
selection strategy.

   This variant of GA provides a new approach to deal with constraint search in laminate composite
optimization, and this method is very easy to extend for solving multiple constraints search problem in other
domain. The problem with the current method is to adjust the parameters in the GA to obtain the best
performance. 

\bibliographystyle{unsrt}

\bibliography{laminate_design}
\end{document}
