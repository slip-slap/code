\section{Introduction}
Composite materials offer improved strength, stiffness, corrosion resistance,
etc. over conventional materials, and are widely used as alternative materials
for applications in various industries ranging from electronic packaging to golf
clubs, and medical equipment to homebuilding, making aircraft structure to space
vechicles. The stacking sequence and fiber orientation of composite laminates
give the designer additional 'degree of freedom' to tailor the design with
respect to strength or stiffness.  One widely known advange of using composite
material is can significantly reducing the weight of target structure, and many
researchers attempted to improve the efficiency of using composite materail by
mimimizing the thickness\cite { schmit1973optimum, schmit1977optimum,
	fukunaga1991strength, soares1995discrete, le1995improved,
	jayatheertha1996application, wang1996optimum, adali1997minimum,
	correia1997higher, scares1997optimization, abu1998optimum, lombardi1998anti,
	le1998design, sivakumar1998optimum, barakat1999use, richard2000reliability,
moita2000sensitivity, soremekun2001composite, walker2003technique,
di2003multiconstrained, kere2003using}.

In practice, fiber orientations are restricted to a finite set of angles, and ply thickness is a
specific numberic value.  Because the design variables are not continueous, a gradient based
optimization procedure, such as gradient descent method, is not suitable to cope with such problems.
Moreover, gradient based optimization approach is very eazily to get trapped in local minima, and
many local optimum may exist in structural optimization problems. A stochastic optimization, such as
genetic algorithm(GA) and simulated annealing(SA), is able to deal with optimization problem with
discrete variables. Besides, stochastic method could escape from local optimum, and obtain global
optimum.  GA is one of the most reliably stochastic algorithm, which has been widely used in solving
constraint desgin for composite laminate\cite{callahan1992optimum,soremekun2001composite,park2001stacking,walker2003technique,deka2005multiobjective,pelletier2006multi,jadhav2007parametric,kim2007development,park2008improved}. Although GA gains different advantages for solving
discrete problems, many disadvantages exists within this approach. First, the optimization process
of GA parameters, such as the population size, parent population,mutation percentage, etc., is very
tedious; Second, the GA needs to evaluate the objective functions many times to acheive the
optimization, and the compuation cost is very high; the last problem within GA is the premature
convergence. GA consists of five basic parts: the variable coding, selection scheme, crossover
operator, mutation operator and how the constraints are handled.

The first issue when implementing a GA is the representation of design variables, because an
appropriate design representation is crucial to enhance the efficiency of GA. Real value string has
been widely employed in 

Selection scheme plays a critical role in balancing the dilemma of exploration and exploitation
inherented in GA, and various selection methods, for example, roulette wheel, elitist, and tournament
etc., have been proposed to overcome this issue. Both of roulette selection and tournament selection
are well-studied and widely employed in the optimization design of laminated composite due to their
simplicity to code and efficiency for both nonparallel and parallel architectures.


multiple types of crossover operator has been utilized in the optimization design of composite
structures, such as: one-point, two-point, and uniform crossover.



GA is originally proposed for unconstrained optimization. However, in order to deal with constrained
design for composite laminate, some techiques were introduced into the GA. The first method is using
of data structure, special data structure was developed to fulfils the symmetry constraint of the
laminate, which consists of coding only half of the laminate and considering that each stack of the
laminate is formed by two laminae with the same orientation but opposite
signs\cite{le1995improved,kogiso1994design}. A penalty function is developed to convert a constrained
problem into an unconstrained problem by adding penalty term to the objective funtion. Another
method to solve constrained problem is introducing repair strategy by Todoroki and Haftka
\cite{todoroki1998stacking}, which is aim to transform infeasible solutions to feasible solution by
incorporating problem-specific knowledge.. 

Another major concern within GA is the convergence speed in terms of the time
and computation cost needed to reach a solution of desired quality. The
objective function based on the CLT is excessively time-consuming and complicate
to evaluate, in addition, the target function of GA  needs to be calculate many
times. The traditional method to deal with this issue is by increasing the
selection pressure to accelerate the convergence speed, however, in some cases,
this approach does not acheive an ideal result. Becasue the GAs just provides a
methodological framework to deal with trickey problems, which is heavily
inspired by evolution of biology, it is unnecessary to exactly follow all the
GA operation. It is possible to just perform one or more GA operations, and
incorporate other techiques into GA. In present study, a variant of mutation
operator is introduced to accelerate the convergence process.
  

To check the feasibility of a laminate composite by imposing a strength
constraint, various failure criterion have been proposed to decide whether it
fails or not, such as  maximum stress failure theory, maximum strain failure
theory, Tsai-Hill Failure theory and Tsai-Wu criterion. Each theory is proposed
based on massive experiment data or complicate mathematical model, however
single use any of them may lead to false optimum design for some loading case
due to the particular shape of its failure envelope. In order to overcome this
disadvantage within every failure theory, two reliably failure criteria, maximum
stress theory and Tsai-wu criterion are employed to check whether the composite
laminate fullfils the constraint.



The rest of the paper is organized as follows. Section 2 explains the classical laminate theory and
the failure criteria taken in the present study.  Section 3 explains the proposed method of
selection strategy and self-adaptative parameters for mutation during the GA process. Section 4
describes the result of the numerical experiments in different cases, and in the Conclusion section
we dicuss the results.






