\section{Introduction}
Composite materials offer improved strength, stiffness, corrosion resistance, etc. over conventional
materials, and are widely used as alternative materials for applications in various industries
ranging from electronic packaging to golf clubs, and medical equipment to homebuilding, making
aircraft structure to space vechicles. One widely known advange of using composite material is can
significantly reducing the weight of target structure.  


thickness\cite { schmit1973optimum, schmit1977optimum, fukunaga1991strength, soares1995discrete,
	le1995improved, jayatheertha1996application, wang1996optimum, adali1997minimum,
	correia1997higher, scares1997optimization, abu1998optimum, lombardi1998anti, le1998design,
	sivakumar1998optimum, barakat1999use, richard2000reliability, moita2000sensitivity,
soremekun2001composite, walker2003technique, di2003multiconstrained, kere2003using}



Genetic
algorithm(GA)\cite{callahan1992optimum,soremekun2001composite,park2001stacking,walker2003technique,deka2005multiobjective,pelletier2006multi,jadhav2007parametric,kim2007development,park2008improved,}

Colony Optimization\cite{aymerich2006ant,}



In practice, fiber orientations are restricted to a finite set of angles, and ply thickness is a
specific numberic value.  Because the design variables are not continueous, a gradient based
optimization procedure, such as gradient descent method, is not suitable to cope with such problems.
Moreover, gradient based optimization approach is very eazily to get trapped in local minima, and
many local optimum may exist in structural optimization problems. A stochastic optimization, such as
genetic algorithm(GA) and simulated annealing(SA), is able to deal with optimization problem with
discrete variables. Besides, stochastic method could escape from local optimum, and obtain global
optimum.  GA is one of the most reliably stochastic algorithm, which has been widely used in solving
constraint desgin for composite laminate. Although GA gains different advantages for solving
discrete problems, many disadvantages exists within this approach. First, the optimization process
of GA parameters, such as the population size, parent population,mutation percentage, etc., is very
tedious; Second, the GA needs to evaluate the objective functions many times to acheive the
optimization, and the compuation cost is very high; the last problem within GA is the premature
convergence. GA consists of five basic parts: the variable coding, selection scheme, crossover
operator, mutation operator and how the constraints are handled.

The first issue when implementing a GA is 

GA is originally proposed for unconstrained optimization. However, in order to deal with constrained
design for composite laminate, some techiques were introduced into the GA. The first method is using
of data structure, special data structure was developed to fulfils the symmetry constraint of the
laminate, which consists of coding only half of the laminate and considering that each stack of the
laminate is formed by two laminae with the same orientation but opposite
signs\cite{le1995improved,kogiso1994design}. A penalty function is developed to convert a constrained
problem into an unconstrained problem by adding penalty term to the objective funtion. Another
method to solve constraint problem is introducing repair strategy by Todoroki and Haftka
\cite{todoroki1998stacking}, which is aim to transform infeasible solutions to feasible solution. 


To check the feasibility of a laminate composite by imposing a strength constraint, various failure
criterion have been proposed to decide whether it fails or not. Rankine and Tresca invented maximum
normal stress theory and maximum shearing stress theory, respectively. It is know as Maximum stress
theory.  St. Venant and Tresca proposed maximum normal strain theory and maximum shear theory,
respectively, also know as Maximum strain failure theory. Tsai-Hill Failure theory is came up based
on the distortion energy failure theory of Von-Mises's distortional energy yield criterion for
isotropic materials. 


The rest of the paper is organized as follows. Section 2 explains the classical laminate theory and
the failure criteria taken in the present study.  Section 3 explains the proposed method of
selection strategy and self-adaptative parameters for mutation during the GA process. Section 4
describes the result of the numerical experiments in different cases, and in the Conclusion section
we dicuss the results.



\section{Analysis of }

\subsection{Stress and Strain in a Lamina}
For a single lamina has a small thickness under plane stress, and it's upper and lower surfaces of the lamina are
free from external loads. According to the Hooke's Law, the three-dimensional stress-strain equations can be reduced to
two-dimensional stress-strain equations. The stress-strain relation in local axis 1-2 is:
\begin{equation}
    \begin{bmatrix}
        \sigma _1\\
        \sigma _2\\
        \tau_{12}
    \end{bmatrix}
    =
    \begin{bmatrix}
        Q_{11} & Q_{12} & 0\\
        Q_{12} & Q_{22} & 0\\
        0      &  0     & Q_{66}
    \end{bmatrix}
    \begin{bmatrix}
        \varepsilon_1\\
        \varepsilon_2\\\gamma_{12}
    \end{bmatrix}
\end{equation}
where $Q_{ij} $are the stiffnesses of the lamina that are related

to engineering elastic constants given by
\begin{equation}
    \begin{split}
    &Q_{11}=\frac{E_1}{1-v_{12}v_{21}}\\
    &Q_{22}=\frac{E_2}{1-v_{12}v_{21}}\\
    &Q_{66}=G_{12}\\
    &Q_{12}=\frac{v_{21}E_2}{1-v_{12}v_{21}}\\
    \end{split}
\end{equation}

where $E_1, E_2, v_{12}, G_{12} $ are four independent engineering elastic constants, which are defined as follows: $E_1 $ is the longitudinal Young's modulus, $E_2 $ is the transverse Young's modulus, $v_{12} $ is the major Poisson's ratio, and $G_{12} $ is the in-plane shear modulus.

Stress strain relation in the global x-y axis:
\begin{equation}\left[\begin{array}{l}\sigma _{x} \\ \sigma _{y} \\ \tau_{xy}\end{array}\right]=\left[\begin{array}{lll}\bar{Q}_{11} & \bar{Q}_{12} & \bar{Q}_{16}\\ \bar{Q}_{12} & \bar{Q}_{22} & \bar{Q}_{26} \\ \bar{Q}_{16} & \bar{Q}_{26} &\bar{Q}_{66}\end{array}\right]\left[\begin{array}{l}\varepsilon_{x} \\ \varepsilon_{y}\\ \gamma_{x y}\end{array}\right]
\end{equation}
where

\begin{equation}
	\begin{array}{l}
		\resizebox{.35\textwidth}{!}{$\bar{Q}_{11}=Q_{11} c^{4}+Q_{22} s^{4}+2\left(Q_{12}+2
		Q_{66}\right) s^{2} c^{2}$} \\

		\resizebox{.35\textwidth}{!}{$\bar{Q}_{12}=\left(Q_{11}+Q_{22}-4 Q_{66}\right) s^{2}
		c^{2}+Q_{12}\left(c^{4}+s^{2}\right)$} \\

		\resizebox{.35\textwidth}{!}{$\bar{Q}_{22}=Q_{11} s^{4}+Q_{22} c^{4}+2\left(Q_{12}+2
		Q_{66}\right) s^{2} c^{2}$} \\

		\resizebox{.4\textwidth}{!}{$\bar{Q}_{16}=\left(Q_{11}-Q_{12}-2 Q_{66}\right) c^{3} s-\left(Q_{22}-Q_{12}-2Q_{66}\right) s^{3} c$}
		 \\ 
		\resizebox{.4\textwidth}{!}{$\bar{Q}_{26}=\left(Q_{11}-Q_{12}-2 Q_{66}\right) c s^{3}-\left(Q_{22}-Q_{12}-2 Q_{66}\right)c^{3} s$}
		 \\ 
	\resizebox{.4\textwidth}{!}	{$\bar{Q}_{66}=\left(Q_{11}+Q_{22}-2 Q_{12}-2 Q_{66}\right)
	s^{2}c^{2}+Q_{66}\left(s^{4}+c^{4}\right)$}\\
	\end{array}
\end{equation}


The c and s denotes $cos\theta $ and $sin\theta $.

The local and global stresses in an angle lamina are related

to each other through the angle of the lamina $\theta $
\begin{equation}\left[\begin{array}{l}\sigma _{1} \\ \sigma _{2} \\ \tau_{12}\end{array}\right]=[T]\left[\begin{array}{l}\sigma _{x} \\ \sigma _{y} \\\tau_{xy}\end{array}\right]
\end{equation}

where
\begin{equation}[T]=\left[\begin{array}{ccc}c^{2} & s^{2} & 2 s c \\ s^{2} & c^{2} & -2 s c \\ -s c & s c &c^{2}-s^{2}\end{array}\right] 
\end{equation}



\subsection{Stress and Strain in a Laminate}
\begin{equation} \label{eq:force_and_moments}
	\begin{array}{l}
		\begin{aligned}
	\begin{bmatrix}
		N_x \\
		N_y \\
		N_{xy}
	\end{bmatrix}
	&=
	\begin{bmatrix}
		A_{11} & A_{12} & A_{16} \\
		A_{12} & A_{22} & A_{26} \\
		A_{16} & A_{26} & A_{66} 
	\end{bmatrix}
    \begin{bmatrix}
		\varepsilon_x^0 \\
        \varepsilon_y^0 \\
		\gamma_{xy}^0
    \end{bmatrix}   \\
	&+               
	\begin{bmatrix}
		B_{11} & B_{12} & B_{16} \\
		B_{11} & B_{12} & B_{16} \\
		B_{16} & B_{26} & B_{66} 
	\end{bmatrix}
	\begin{bmatrix}
		k_x \\
		k_y \\
		k_{xy} 
	\end{bmatrix}  \\
\end{aligned} \\ \\
\begin{aligned}
	\begin{bmatrix}
		M_x \\
		M_y \\
		M_{xy}
	\end{bmatrix}
	&=
	\begin{bmatrix}
		B_{11} & B_{12} & B_{16} \\
		B_{12} & B_{22} & B_{26} \\
		B_{16} & B_{26} & B_{66} 
	\end{bmatrix}
    \begin{bmatrix}
		\varepsilon_x^0 \\
        \varepsilon_y^0 \\
		\gamma_{xy}^0
    \end{bmatrix} \\ 
	&+  
	\begin{bmatrix}
		D_{11} & D_{12} & D_{16} \\
		D_{11} & D_{12} & D_{16} \\
		D_{16} & D_{26} & D_{66} 
	\end{bmatrix}
	\begin{bmatrix}
		k_x \\
		k_y \\
		k_{xy} 
	\end{bmatrix}
\end{aligned}
	\end{array}
\end{equation}


$N_x,N_y $  - normal force per unit length

$N_{xy} $  - shear force per unit length

$M_x, M_y $ - bending moment per unit length

$M_{xy} $  - twisting moments per unit length

$\varepsilon^{0}, k $- mid plane strains and curvature of a laminate in x-y coordinates

The mid plane strain and curvature is given by
\begin{equation}
    \begin{split}
    &A_{ij}=\sum_{k=1}^{n}(\overline{Q_{ij}})_k(h_k-h_{k-1})  i=1,2,6, j=1,2,6\\
    &B_{ij}=\frac{1}{2}\sum_{k=1}^{n}(\overline{Q_{ij}})_k(h_k^2 - h_{k-1}^2)  i=1,2,6, j=1,2,6\\
    &D_{ij}=\frac{1}{3}\sum_{k=1}^{n}(\overline{Q_{ij}})_k(h_k^3 - h_{k-1}^3) i=1,2,6, j=1,2,6\\
    \end{split}
\end{equation}

The [A], [B], and [D] matrices are called the extensional, coupling, and bending stiffness matrices,
respectively. The extensional stiffness matrix $[A]$ relates the resultant in-plane forces to the
in-plain strains, and the bending stiffness matrix $[D]$ couples the resultant bending moments to
the plane curvatures.  The coupling stiffness matrix $[B]$ relates the force and moment terms to the
midplain strains and midplane curvatures.


\section{Failure criteria for a lamina}
Many different criteria about the failure of an angle lamina have been proposed for a unidirectional
lamina, such as the maximum stress failure theory, maximum strain failure theory, Tsai-Hill failure
theory, and Tsai-Wu failure theory. The failure criterion of a lamina are based on the stresses in
the local axes instead of principal normal stresses and maximum shear stresses. There are four
normal strength parameters and one shear stress for a unidirectional lamina. The five strength
parameters are

$(\sigma _1^{T})_{ult}= $ ultimate longitudinal tensile strength(in direction 1),

$(\sigma _1^{C})_{ult}= $ ultimate longitudinal compressive strength(in direction 1),

$(\sigma _2^{T})_{ult}= $ ultimate transverse tensile strength(in direction 2),

$(\sigma _2^{C})_{ult}= $ ultimate transverse compressive strength(in direction 2), and

$(\tau_{12})_{ult}= $ and ultimate in-plane shear strength(in plane 12).

In the present study, Tsai-wu failure theory is taken to decide whether a lamina fails,
because this theory is more general than the Tsai-Hill failure theory, which considers two
different situations, the compression and tensile strengths of a lamina. A lamina is considered to fail
if \begin{equation} \label{eq:tsai_wu}
\begin{split}
	H_1 \sigma_1  & + H_2 \sigma_2 + H_6 \tau_{12} + H_{11}\sigma_1^2 + H_{22} \sigma_2^2 \\
				  & + H_{66}  \tau_{12}^2 + 2H_{12}\sigma_1\sigma_2 < 1
\end{split}
\end{equation}

is violated, where

\begin{equation} \label{eq:sr}S R=\frac{\text {Maximum Load Which Can Be Applied}}{\text {Load Applied}}
\end{equation}

Maximum stress failure theory consists of maximum normal stress theory proposed by Rankine and maximum 
shearing stress theory by Tresca. The stresses applied on a lamina can be resolved into the normal and shear stresses 
in the local axes. If any of the normal or shear stresses in the local axes of a lamina is equal or exceeds the corresponding 
ultimate strengths of the unidirectional lamina, the lamina is considered to be failed. 



\subsection{Failure Theories for a Laminate}
If keep increasing the loading applied to a laminate, the laminate will fails. The failure process
of a laminate is more complicate than a lamina, because a laminate consists of multiple plies, and
the fiber orientation, material, thickness of each ply maybe different from the others. In most
situations, some layer fails first and the remains continue to take more loads until all the plies
fail.  If one ply fails, it means this lamina does not contribute to the load carrying capacity of
the laminate. The procedure for finding the first failure ply given follows the fully discounted
method:

\begin{enumerate}
\item Compute the reduced stiffness matrix [Q] referred to as the local axis for each ply using its
	four engineering elastic constants $E_1 $, $E_2 $, $E_{12} $, and $G_{12} $.

\item Calculate the transformed reduced stiffness $[\bar{Q}] $ referring to the global coordinate
	system (x, y) using the reduced stiffness matrix [Q] obtained in step 1 and the ply angle for
	each layer.

\item  Given the thickness and location of each layer, the three laminate stiffness matrices [A],
	[B], and [D] are determined.

\item  Apply the forces and moments, $[N]_{xy}, [M]_{xy} $ solve Equation
	\ref{eq:force_and_moments}, and calculate the middle plane strain $[\sigma ^{0}]_{xy} $ and
	curvature $[k]_{xy} $.

\item Determine the local strain and stress of each layer under the applied load.

\item  Use the ply-by-ply stresses and strains in the Tsai-wu failure theory to find the strength
	ratio, and the layer with smallest strenght ratio is the first failed ply. 
\end{enumerate}

