
\section{Methodology}
\subsection{Objective function}
There are two design variables here, the angles in the laminate, and the number of layers that each
fiber orientation has. The objective function is as

$F  = 2t_0 \sum_{k=1}^n n_k$ 

The first term represent the total thickness of the composite laminates, $t_0$ is ply thickness;
$n_k$ is the number of plies in the kth lamina, in which the fiber orientation is $\theta_k$.

The only constraint is the safety factor of the material under certain loading, and it should
greater than 1.

\subsection{Selection}
The purpose of the selection operator is how to chose parents to produce children of better
fitness. Traditional methods of selecting strategies only take the fitness of the individual into
acount, however, becasue of the existance of constraint, the selection strategies have to change a
little bit. The parents of next generation consists of three groups: proper groups, active groups,
and potential groups. 

Proper parents mean individual fullfils the constraint, which are chosen by the individual's
fitnees, individuals with better fitness are more likely to be chosen if they fit the constraint;
active groups means that individual is supposed to be always exist in the parents during the GA,
which are selected by fitness, ignoring the constraint; potential groups means that they are likely
to turn into proper individual after a couple of generations, and potential individuals are chosen
by constraint function, the more the individual fulfils the constraint, the more possiblity it will
be selected.


	
\subsection{Crossover}
The crossover operator happens among these three groups. the child of two proper groups are more
likely to be a proper individual which can be used to obtain a better individual. the child of an
active individual and a potential individual can significantly change the gene of active
individual's chromsome, which lets the individual evolved toward a new direction.  The offspring of
two active individuals are more likely to be an active individual, which can maitain the active
group.
\subsection{Mutation}
A mutation direction is imposed on the mutation operator which to make sure the individual evolving
toward the right direction. The mutation direction, denoted by $md$, is a n dimensional vector corresonding to
number of constraints, it is decided by the constraint thresholds $CT_i$ and the current individual's
constraint value, denoted as $CV_i$,  The mutation vector can be obtained by the following formula

$\text{md} = [CT_1, \cdots, CT_{n-1}, CT_n] -  [CV_0, \cdots, CV_{n-1}, CV_n]$

During the operator, the mutation consists of three parts, the length of the chromsome, the angle
of the chromsome, and the number of each angle. Becasue the chromsome's length is positive correlated with the individual's
fitness, the coefficient of length mutation denoted by $C_l$, if $\sum_{i=1}^{N}{CT_i}$ great than
zero, the mutation length is restricted to the range $[0,C_l \sum_{i=1}^{N}{CT_i}]$; if the
$\sum_{i=1}^{N}{CT_i}$ less than zero, the mutation length is restricted to the range
$[0,\sum_{i=1}^{N}{CT_i}]$; Assuming a $[13_6/-27_4]_s$ carbon T300/5308 composite laminate under
the loading $N_{xx} = N_{yy} = 10$ MPa m, the only constraint is the safety factor greater than 1.
Accoring to the Tsai-Wu criterion, its safety factor is 0.0539. So the mutation vector is $[0.941]$,
assuming the coefficient is 20, so the mutation range is from 0 to 18. A random number is generated
from the range $[0, 18]$, supposing the outcome is 13, then a length generator is used to a list,
the it's sum is 13, suppose the list is [5, 8], the laminate after mutation is $[13_{11}/-27_{12}]_s$.


The relationship between the angles in the composite laminate and the chromsome's fitness is
unclear, so the mutation direction of chromsome's angle is random. The coefficient angle mutation is
$C_a$, $[0,C_a \sum_{i=1}^{N}{CT_i}]$

